\section{Related Work}
\label{sec:relatedwork}

Self-adaptive middleware frameworks are an emerging area in software engineering. Java and OSGi have been the principal platforms to prototype and test such systems. However, to the best of our knowledge none of the implementations address the problem of reusing legacy software libraries. They also do not use empirical studies to validate their framework. We briefly describe two recent projects.

DIVA (Dynamic Variability in complex, Adaptive systems) is a European project that provide tools and methodologies to manage dynamic variability in adaptive systems. The approach proposes the use of model-driven and aspect-oriented techniques \cite{Romero2010}. DIVA addresses the problem of explosion of possible system configurations and the migration from the current configuration to a valid target configuration \cite{Morin2009} in an adaptive system.  The framework has been develop using the OSGi specification on the Helios Eclipse Equinox implementation. However, DIVA is a heavy-weight framework and performs adaptations in the order of seconds and not milliseconds. DIVA also does not address more intricate issues such as its functionality due  frequency of adaptation. DIVA has been applied to a home automation system called Entimid. Entimid contains a number of OSGi components for sensors, behavior and actuators situated in an apartment for handicapped and elderly people. Entimid tends to behave very erratically when contextual events that trigger adaptations arrive in unpredictable ways. In our opinion, this is primarily due to lack of empirical analysis of the middleware framework taking into account the deterministic/non-deterministic QoS models of the various components.

MUSIC (Self-Adapting Applications for Mobile Users in Ubiquitous Computing Environments)  is an European project  that provides an open source platform for development and execution of self-adaptive systems. MUSIC provides self-adaptive mobile applications for different devices and operating systems. It also provides developer tools that simplify its use. The platform bases its decisions on monitoring and sensing QoS characteristics of the components that compose the running system. With the help of utility or weight functions the system measures how the system can adapt to a specific context. The adaptation framework has a model of the system structured with several components that can be modified at run time. It also uses the event-driven architecture to manage context events; this allows the system to provide loosely coupled software components. The project goal is to maintain self-adaptation separated from the business logic. However, the solution does not consider the multiple numbers of possible variations, adaptation frequency, and the use of legacy libraries.
