\section{Introduction}
\label{sec:introduction}

Self-adaptive systems dynamically modify themselves due to contextual changes and feedback from within their own components \cite{Oreizy}. Contextual changes may emanate from monitoring events from the physical environment, running software/hardware components, and detectors of social and lingual boundaries to name a few. Moreover, these software systems continue running  despite user interventions and failures in the underlying software and hardware \cite{Brun2009}. Well known examples of such systems are sites such as Facebook and Google.  Contrary to these modern systems legacy software systems/libraries were not built with continual execution, adaptation to context and fault tolerance in mind. Therefore, a natural question arises: Can we reuse existing legacy libraries as components in a self-adaptive framework where they can be loaded/unloaded/replaced at runtime?

In this paper, we address the challenge of reusing legacy open source libraries in our bare-minimum self-adaptive middleware framework called Tekio. Tekio adheres to the requirements in  \cite{Hallsteinsen2006} for component frameworks to implement dynamic self-adaptive systems. It has the following functionality:(1) Component management that helps define components and the interactions amongst them, (2) Instance Management that permits the component life-cycle to be administered and (3)Self-Adaptation management for context understanding and mapping context to system. Tekio is a implemented in Java. Self-adaptation in Tekio is achieved using \emph{dynamic component loading}  provided by the  OSGi framework specification (formerly known as the Open Services Gateway Initiative). The OSGi framework has applications ranging from mobile applications, IDE, applications servers to software in automobile industries. The OSGi has 136 official members plus several research projects. It has seven implementations such as Eclipse Equinox, Apache felix, Knopflerfish and projects such as JBoss, Glasfish Fuse EXB Eclipse platform and WebSphere. The widely used OSGi provides the basic functionality to create self-adaptive systems. This paper evaluates OSGi as the basis to realize self-adaptive middleware to reuse legacy software such as Tekio.

We use Tekio to build a self-adaptive vision system. This system serves as a case study to evaluate Tekio and OSGi as the middleware platform to reuse legacy open-source libraries. We reuse the OpenCV libraries in software components dynamically managed by Tekio. Tekio components call native code in C/C++ using Java Native Access. We provide number of configurations of these components for adaptation. These configurations achieve tasks such as intrusion detection, face detection, and segmentation. During adaptations we measure frames per second indicating throughput. We also measure the settling time between adaptations. Settling time indicates the time required by Tekio to produce meaningful outputs after adaptation. We perform experiments to demonstrate that Tekio's throughput for a configuration is very close to an identical native implementation despite the self-adaptation layer. The self-adaptive system can demonstrate very low settling times for low and medium resolution input video. For instance, it can provide about 30 adaptations in 2 seconds without significant loss in throughput. However, for high resolution input videos the system cannot adapt is allowed to adapt less frequently to provide meaningful outputs. From these results we incur that managing  self-adaptation requires rigorous empirical analysis and may entail trade-offs.

The paper is organized as follows. In Section \ref{sec:relatedwork}, we present background material. In Section \ref{sec:architecture}, we present Tekio's architecture based on OSGi. In Section \ref{sec:validation}, we validate Tekio. In Section \ref{sec:conclusion}, we conclude.