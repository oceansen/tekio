\section{Conclusion}
\label{sec:conclusion}

In this paper, we demonstrate that legacy software libraries  can be cast into a self-adaptive middleware framework: Tekio. Tekio is based on the lightweight OSGi standard for dynamic component loading that facilitates self-adaptation of functions in legacy libraries. Using the computer vision library OpenCV we demonstrate that Tekio can be used to build a self-adaptive vision system. We evaluate Tekio for a number of configurations of a vision system. First, we demonstrate that Tekio's performance is negligibly slower for a fixed configuration compared to an identical implementation in native C. The OSGi layer and Tekio's middleware do not incur a large performance overhead. Second, we demonstrate that Tekio can handle about 30 adaptations in a span of 2 seconds. This result however, is dependent on the input video resolution. Only low/medium resolution videos can be dealt with despite high rates of adaptation. When high resolution videos are treated the 30 adaptations may occur in the span of at least 90 seconds. If the adaptation rate is too high Tekio simply stops producing output. It does not crash which open doors to techniques for self-healing.

As future work we would like Tekio  to provide self-healing and self-protection to the running system. For example, Tekio must be provide the possibility of cleaning up unused memory. Resource management including memory management and debugging of legacy components from Tekio is an important goal of our future work. We would also like Tekio to perform  selection from two or more  candidates configurations for an adaptation.  Tekio also needs to maintain a model@runtime to simplify the interface to reconfiguration in Tekio. 